\documentclass[conference]{IEEEtran}
\IEEEoverridecommandlockouts
\renewcommand{\refname}{Bilbiografía}
% The preceding line is only needed to identify funding in the first footnote. If that is unneeded, please comment it out.
\usepackage{cite}
\usepackage{amsmath,amssymb,amsfonts}
\usepackage{algorithmic}
\usepackage{graphicx}
\usepackage{textcomp}
\usepackage{xcolor}
\usepackage{url}
\def\BibTeX{{\rm B\kern-.05em{\sc i\kern-.025em b}\kern-.08em
    T\kern-.1667em\lower.7ex\hbox{E}\kern-.125emX}}
\begin{document}

\title{Tarea Corta 1\\
{\footnotesize \textsuperscript{}IC3002 - Análisis de Algoritmos}
{\footnotesize \textsuperscript{}Profe.: Yuen Law Wan}
{\footnotesize \textsuperscript{}I Semestre 2020}
}

\author{\IEEEauthorblockN{Joseph Tenorio Pereira}
\IEEEauthorblockA{\textit{2019064588} \\
}
\and
\IEEEauthorblockN{Jose Pablo Muñoz}
\IEEEauthorblockA{\textit{2019061904} \\
}
}

\maketitle

\begin{abstract}
El abstract va aqui
\end{abstract}


\section{Introducción}
La introducción va aqui\\

\section{Descripción de los algoritmos}
A continuación se dará una breve explicación de los algoritmos de ordenamiento analizados por medio
de la graficación en \textit{Unity}. Estos corresponden a los métodos de \textit{Bubble Sort} y \textit{Quick Sort}, programados
en el lenguaje \textit{C\#}. En dicha graficación, se estableció una relación entre la cantidad de elementos a ordenar y
 la cantidad de microsegundos que le toma al algoritmo el realizar dicho ordenamiento. 


\subsection{Agoritmo de odenamiento Bubble Sort}

El primer método de ordenamiendo analizado corresponde al ordenamiento de burbuja, mejor conocido como \textit{Bubble Sort}.
Este método es una simple e intuitiva solución al problema de acomodar ascendentemente una serie de números. 
Dicha solución consiste en comparar cada número del arreglo con el número siguiente
 (excepto el último número), e intercambiarlos si están en el orden incorrecto. Dicho recorrido, con 
sus respectivas comparaciones e intercambios, se realiza una vez por cada elemento del arreglo
 (como es el caso del algoritmo empleado en este trabajo),o bien, hasta que se realiza un recorrido completo sin intercambios.

\subsection{Agoritmo de odenamiento Quick Sort}

Por otro lado, el segundo método de ordenamiento analizado corresponde al ordenamiento rápido, mejor conocido
como \textit{Quick Sort}. Este algoritmo pertenece a la categoría de divide y vencerás, en inglés, \textit{Divide and Conquer},
ya que selecciona un elemento del arreglo como pivote y particiona los elementos de dicho arreglo en mayores y menores que el 
pivote, posteriormente se ubica el pivote en la poisición correcta\cite{b1}. El proceso anterior es luego repetido recursivamente sobre
los arreglos resultantes de la partición, hasta que todos ellos queden ordenados. El algoritmo empleado en el presente 
análisis toma como pivote al primer elemento de los arreglos y fue tomado de \cite{b1}.\\

\section{Análisis de resultados}
Como ya se mencionó anteriormente, el gráfico empleado para este trabajo, apreciable en [Fig. 1], establece un relación entre
 el número de elementos a ordenar (eje X) y el tiempo de ejecución (eje Y, en microsegundos) de los algoritmos de
 \textit{Bubble Sort} y \textit{Quick Sort}. Cabe resaltar que para cada tamaño de arreglo del eje X se realizaron cinco corridas de
los algoritmos, representadas por los puntos verdes y celestes según sea el caso.

\begin{figure}[htbp]
\centerline{\includegraphics[scale=0.34]{grafico.png}}
\caption{Gráfico resultante en \textit{Unity}.}
\label{fig}
\end{figure}

El primer detalle a notar es el hecho de que el algoritmo de \textit{Bubble Sort} se vuelve extremedamente ineficiente conforme
aumenta el tamaño del arreglo a ordenar, en especial si lo comparamos con el algoritmo de \textit{Quick Sort}. De hecho, la graficación
podría dar a entender que el algoritmo de \textit{Quick Sort} mantiene un tiempo de ejecución constante, pero este no es el caso. En realidad,
el crecimiento del tiempo de ejecución del algoritmo de \textit{Quick Sort} no se puede observar en las escalas requeridas para representar
el crecimiento exponencial del tiempo de ejecución del algoritmo de \textit{Bubble Sort}. Lo anterior se debe a que el metodo \textit{Quick Sort}
presenta un crecimiento de tiempo de ejecución logarítmico, el cual, nuevamente, es inapreciable debido a la escala usada.
Otro detalle a notar en los resultados obtenidos es que los tiempos de ejecución del algoritmo de \textit{Bubble Sort} empleado poseen 
una mayor varianza conforme crece el tamaño del arreglo ordenado.

\section{Acceso al proyecto}
El repositorio de GitHub en el cual se ubica el proeycto de \textit{Unity} empleado puede ser accesado mediante el siguiente link: \url{https://github.com/TilapiaBoi/Analisis_TC1.git}

\begin{thebibliography}{00}
\bibitem{b1} W3resource.com, ``C\# Sharp Searching and Sorting Algorithm Exercises: Quick sort'', Nov. 2019. Accesado el: 21 Feb., 2020. [Online]. Disponible en: https://www.w3resource.com/csharp-exercises/searching-and-sorting-algorithm/searching-and-sorting-algorithm-exercise-9.php
\end{thebibliography}
\end{document}